
%\documentclass[11pt]{article} %指定文档的类型和基本格式。这里选择了article类,字体大小为11磅。
\documentclass[UTF8]{ctexart}
\usepackage{amsmath,textcomp,amssymb,geometry,graphicx,enumerate} %加载了一些宏包,这些宏包提供了额外的功能和格式设定,例如数学符号、文本特殊符号、排版布局等。

\usepackage{listings}
\usepackage[dvipsnames]{xcolor}

%\usepackage[utf8]{inputenc}
%\usepackage{xeCJK} % Added for Chinese support
%\setCJKmainfont{SimSum} % Set the Chinese font, you can change it to any font you have


%\def\Name{zilong}
\def\Name{王子隆}  % Your name 定义了一个名为\Name的变量,它的内容是"PUT SOMETHING HERE"。通常这个变量会在文档中被调用以插入相应的信息。
\def\SID{2221411126}  % Your student ID number 定义了一个名为\SID的变量,它的内容也是"PUT SOMETHING HERE"。
\def\Homework{1} % Number of Homework 定义了一个名为\Homework的变量,它的内容是"n",表示作业编号。
\def\Session{Autumn 2023} %定义了一个名为\Session的变量,它的内容是"Autumn 2023",表示学期。


\title{DS--Autumn 2023 --- Homework \Homework Solutions} %设置了文档的标题,包括课程名、学期和作业编号。
\author{\Name, SID \SID} %设置了文档的作者信息,使用了之前定义的\Name和\SID变量。
\markboth{DS--\Session\  Homework \Homework\ \Name}{DS--\Session\ Homework \Homework\ \Name} %设置了页眉的内容。
\pagestyle{myheadings} %设置了页眉的内容。
\date{\today} %清空了默认的日期显示。

\newenvironment{qparts}{\begin{enumerate}[{(}a{)}]}{\end{enumerate}} %定义了一个新的环境qparts,用于创建带有小括号标记的题目部分。
%\def\endSolutionsmark{$\mathcal{X} \mathcal{I} \mathcal{U} $} %定义了证明结束标记为一个方框符号。
\newenvironment{Solutions}{\par{\bf Solutions}:}%{\endSolutionsmark\smallskip} %定义了一个新的环境Solutions,用于书写数学证明。


\lstdefinestyle{mystyle}{ %%定义了一个样式mystyle,指定了Java语言,设定了代码的基本样式、注释样式、关键词样式等等
    language=Java,
    basicstyle=\ttfamily\small, % 设置代码字体为大一些
    commentstyle=\color{OliveGreen},
    keywordstyle=\color{RoyalBlue},
    numberstyle=\tiny\color{gray},
    numbers=left,
    frame=single,
    breaklines=true,
    breakatwhitespace=true,
    tabsize=4
}


\textheight=9in
\textwidth=6.5in
\topmargin=-.75in
\oddsidemargin=0.25in
\evensidemargin=0.25in %设置了页面的尺寸和边距。


\lstset{style=mystyle}

\begin{document}
\maketitle

%Collaborators: PUT SOMETHING HERE (LIST OF YOUR COLLABORATORS, OR WRITE NONE)

\begin{abstract}
    %Abstract goes here...
    线性表功能实现及应用
    
    Merge(L1,L2)

    Dispose (L1, 12)

    Sort (L)
  
    Insert(L,a,b) 不/带头链表L,在元素值a之前插入b

    应用 一元多项式的加減法(链表)


    %插入元素通过覆盖元素,并且向后增加节点

    检测环形链表证明链表有无环

    增加bool数据类型判断环形链表是否遍历完毕

    hanoi塔

    数据库(交并差叉积)
    
    半矩阵存储

\end{abstract}













\section*{1. 线性表功能实现及应用}
\begin{qparts}
    \item 
    功能实现
    
    Merge(L1,L2)

    Dispose (L1, 12)

    Sort (L)
  
    Insert(L,a,b) 不/带头链表L,在元素值a之前插入b

    \item
    应用 
    
    一元多项式的加減法(链表)
\end{qparts}

\subsection*{1.1 数组、链表、双向循环链表建立}

\begin{Solutions}
\begin{enumerate} % for numbers

\item 数组
ArrayDeque.java
\begin{lstlisting}
/** Array based list.
*  @author zilong
*/

/* Invariants:
addLast: The next item we want to add, will go into position size
getLast: The item we want to return is in position size - 1
size: The number of items in the list should be size.
*/
public class ArrayDeque<T> implements List<T>{
    private T[] items;
    private int size;

    /** Creates an empty array deque. */
    public ArrayDeque() {
        items = (T[]) new Object[8];
        size = 0;
    }

    /** Resizes the underlying array to the target capacity. */
    private void resize(int capacity) {
        T[] a = (T[]) new Object[capacity];
        System.arraycopy(items, 0, a, 0, size);
        items = a;
    }
    /** Adds an item of type T to the front of the deque.*/
    public void addFirst(T item) {
        if (size == items.length) {
            //this.resize(size + 1);
            this.resize(size + 1);
        }
        //for (int i = size; i > 0; i--) {
        //    items[i] = items[i -1 ];
        //}
        System.arraycopy(items, 0, items, 1, size);
        items[0] = item;
        size = size + 1;
    }

    /** Inserts X into the back of the list. */
    public void addLast(T item) {
        if (size == items.length) {
            //this.resize(size + 1);
            this.resize(size + 1);
        }
        items[size] = item;
        size = size + 1;
    }


    /** Returns true if deque is empty, false otherwise.*/
    public boolean isEmpty() {
        if (size == 0) {
            return true;
        }
        return false;
    }

    /** Prints the items in the deque from first to last, separated by a space.
    * Once all the items have been printed, print out a new line.*/
    public void printDeque() {
        if (size == 0) {
            System.out.println("The deque is empty.\n");
        }
        for (int i = 0; i < size; i++) {
            System.out.print(items[i] + " ");
        }
        System.out.print("\n");
    }

    /** Gets the ith item in the list (0 is the front).
    * If no such item exists, returns null. Must not alter the deque! */
    public T get(int i) {
        return  items[i];
    }

    public T getFirst() {
        return get(0);
    }

    public T getLast() {
        int lastActualItemIndex = size - 1;
        return items[lastActualItemIndex];
    }

    /** Returns the number of items in the list. */
    public int size() {
        return size;
    }

    /** Removes and returns the item at the front of the deque.
    * If no such item exists, returns null. */
    public T removeFirst() {
        if (size == 0) {
            return null;
        }
        T front = get(0);


        //for (int i = 0; i < size - 1; i++) {
        //    items[i] = items[i + 1];
        //}

        System.arraycopy(items, 1, items, 0, size - 1);

        items[size - 1] = null; //this may not right.

        size = size - 1;
        if (size <= items.length * 0.25 && size > 16) {
            resize((int) (items.length * 0.25));
        }
        return front;
    }

    /** Deletes item from back of the list and
    * returns deleted item. */
    public T removeLast() {
        if (size == 0) {
            return null;
        }
        T back = get(size - 1);
        items[size - 1] = null;
        size = size - 1;
        if (size <= items.length * 0.25 && size > 16) {
            resize((int) (items.length * 0.25)+1);
        }
        return back;
    }


    /** Inserts item into given position.*/
    public void insert(T x, int position) {
        T[] newItems = (T[]) new Object[items.length + 1];

        System.arraycopy(items, 0, newItems, 0, position);
        newItems[position] = x;

        System.arraycopy(items, position, newItems, position + 1, items.length - position);
        items = newItems;
    }


}

\end{lstlisting}

\item 链表
Intlist.Java
\begin{lstlisting}
    public class IntList {

    private int first;
    private IntList rest;

    /** A List with first FIRST0 and rest REST0. */
    public IntList(int first0, IntList rest0) {
        first = first0;
        rest = rest0;
    }

    /** A List with null rest, and first = 0.*/
    public IntList() {
        /* NOTE: public IntList () { }  would also work. */
        this(0, null);
    }

    
    /**
    * Returns a list consisting of the elements of A followed by the
    * *  elements of B.  May modify items of A.
    */

    public static IntList dmerge(IntList A, IntList B) {
        if (A == null) {
            A = B;
            return A;
        }
        IntList prt = A;
        while (prt.rest != null) {
            prt = prt.rest;
        }
        prt.rest = B;
        return A;
    }

    /**
    * Returns a list consisting of the elements of A followed by the
    * * elements of B.  May NOT modify items of A. 
    */

    public static IntList merge(IntList A, IntList B) {
        if (A == null) {
            return B;
        }
        if (A.rest == null) {
            return new IntList(A.first, B);
        }
        return new IntList(A.first, merge(A.rest, B));
    }
}
\end{lstlisting}

\item 双向循环链表
LinkedListDeque.java
\begin{lstlisting}
/** DLLists based list.
*  @author zilong
*/

/* Invariants:
addLast: The next item we want to add, will go into position size
getLast: The item we want to return is in position size - 1
size: The number of items in the list should be size.
*/

public class LinkedListDeque<T> implements List<T>{

    /* Double-ended queues are sequence containers with dynamic sizes
    that can be expanded or contracted on both ends (either its front or its back).*/

    private class StuffNode {
        private StuffNode prev;
        private T item;
        private StuffNode next;
        public StuffNode(StuffNode f, T i, StuffNode n) {
            prev = f;
            item = i;
            next = n;
        }
        /** Returns the ith item of this IntList. */
        public T get(int i) {
            if (i == 0) {
                return item;
            }
            return next.get(i - 1);
        }
    }

    /* The first item (if it exists) is at sentinel.next. */
    private StuffNode sentinel;
    private int size;

    public LinkedListDeque() {
        size = 0;
        sentinel = new StuffNode(null, null, null);
        sentinel.next = sentinel;
        sentinel.prev = sentinel;
    }

    public LinkedListDeque(T x) {
        sentinel = new StuffNode(null, null, null);
        sentinel.next = new StuffNode(sentinel,x,sentinel);
        sentinel.prev = sentinel.next;
        size = 1;
    }

    /** Creates a deep copy of other.*/
    /*public LinkedListDeque(LinkedListDeque other) {
        size = 0;
        sentinel = new StuffNode(null, null, null);
        sentinel.next = sentinel;
        sentinel.prev = sentinel;
        for (int i = 0; i < other.size(); i++) {
            this.addLast((T) other.get(i + 1));
        }
    }*/

    /** Adds an item of type T to the front of the deque.*/
    public void addFirst(T item) {
        size = size + 1;
        sentinel.next = new StuffNode(sentinel, item, sentinel.next);
        if (size == 1) {
            sentinel.prev = sentinel.next;
        }
        sentinel.next.next.prev = sentinel.next;
    }

    /** Adds an item of type T to the back of the deque.*/
    public void addLast(T item) {
        size = size + 1;
        sentinel.prev.next = new StuffNode(sentinel.prev, item, sentinel);
        sentinel.prev = sentinel.prev.next;
    }

    /** Returns true if deque is empty, false otherwise.*/
    public boolean isEmpty() {
        /* if (sentinel.next.item == null){
            return true;
        }
        return false;*/
        if (size == 0) {
            return true;
        }
        return false;
    }

    /**  Returns the number of items in the deque.*/
    public int size() {
        return size;
    }
    /** Prints the items in the deque from first to last, separated by a space.
    * Once all the items have been printed, print out a new line.*/
    public void printDeque() {
        StuffNode p = sentinel.next;
        if (size == 0) {
            System.out.println("The deque is empty.");
        }
        for (int i = 0; i < size; i++) {
            System.out.print(p.item + " ");
            p = p.next;
        }
        System.out.println("\n");
    }

    /**Removes and returns the item at the front of the deque.
    * If no such item exists, returns null.*/
    public T removeFirst() {
        if (size == 0) {
            return null;
        }
        T front = sentinel.next.item;
        sentinel.next.next.prev = sentinel;
        sentinel.next = sentinel.next.next;
        size = size - 1;
        return front;
    }

    /** Removes and returns the item at the back of the deque.
    * If no such item exists, returns null.*/
    public T removeLast() {
        if (size == 0) {
            return null;
        }
        T back = sentinel.prev.item;
        sentinel.prev.prev.next = sentinel;
        sentinel.prev = sentinel.prev.prev;
        size = size - 1;
        return back;
    }

    /** Gets the item at the given index, where 0 is the front, 1 is the next item, and so forth.
    * If no such item exists, returns null. Must not alter the deque!*/
    public T get(int index) {
        StuffNode p = sentinel;
        if (index > size) {
            return null;
        }
        for (int i = 0; i <= index; i++) {
            p = p.next;
        }
        //System.out.println(p.item);
        return p.item;
    }

    
    public T getFirst() {
        return get(0);
    }

    public T getLast() {
        return get(size - 1);
    }


    /** Same as get, but uses recursion.*/
    public T getRecursive(int index) {
        if (index > size) {
            return null;
        }
        return sentinel.next.get(index);
    }

    
    /** Inserts item into given position.*/
    public void insert(T item, int position) {
        if (sentinel.next == null || position == 0) {
            addFirst(item);
            return;
        }

        StuffNode currentNode = sentinel.next.next;
        while (position > 1 && currentNode.next != null) {
            position -= 1;
            currentNode = currentNode.next;
        }

        StuffNode newNode = new StuffNode(currentNode, item, currentNode.next);
        currentNode.next = newNode;
    }	

}

\end{lstlisting}

\end{enumerate}

\end{Solutions}


\subsection*{1.2 应用 }

\begin{Solutions}
\begin{enumerate}
\item interface 
List.java
\begin{lstlisting}
public interface List<Item> {
    /**
     * Inserts X into the back of the list.
     */
    public void addLast(Item x);

    /**
     * Returns the item from the back of the list.
     */
    public Item getLast();

    /**
     * Gets the ith item in the list (0 is the front).
     */
    public Item get(int i);

    /**
     * Returns the number of items in the list.
     */
    public int size();

    /**
     * Deletes item from back of the list and
     * returns deleted item.
     */
    public Item removeLast();

    /**
     * Inserts item into given position.
     * Code from discussion #3
     */
    public void insert(Item x, int position);

    /**
     * Inserts an item at the front.
     */
    public void addFirst(Item x);

    /**
     * Gets an item from the front.
     */
    public Item getFirst();

    /** Prints the list. Works for ANY kind of list. */
    default public void print() {
        for (int i = 0; i < size(); i = i + 1) {
            System.out.print(get(i) + " ");
        }
    }
}
\end{lstlisting}

\end{enumerate}

\end{Solutions}

\newpage

\iffalse
\section*{2.}
\begin{qparts}
\item
YOUR ANSWER GOES HERE

\end{qparts}


\newpage
\section*{3.}
YOUR ANSWER GOES HERE


\newpage
\section*{4.}
YOUR ANSWER GOES HERE


\newpage
\section*{5.}
YOUR ANSWER GOES HERE


\newpage
\section*{6.}
YOUR ANSWER GOES HERE

\fi
\end{document}
