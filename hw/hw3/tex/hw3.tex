\documentclass[UTF8]{ctexart}
%\documentclass[11pt]{article} %指定文档的类型和基本格式。这里选择了article类,字体大小为11磅。
\usepackage{amsmath,textcomp,amssymb,geometry,graphicx,enumerate} %加载了一些宏包,这些宏包提供了额外的功能和格式设定,例如数学符号、文本特殊符号、排版布局等。

\usepackage{listings}
\usepackage[dvipsnames]{xcolor}

%\usepackage[utf8]{inputenc}
%\usepackage{xeCJK} % Added for Chinese support
%\setCJKmainfont{SimSum} % Set the Chinese font, you can change it to any font you have


%\def\Name{zilong}
\def\Name{王子隆}  
\def\SID{2221411126}  
\def\Homework{3} 
\def\Session{Autumn 2023} 


\title{DS--Autumn 2023 --- Homework \Homework Solutions} 
\author{\Name, SID \SID} 
\markboth{DS--\Session\  Homework \Homework\ \Name}{DS--\Session\ Homework \Homework\ \Name} 
\pagestyle{myheadings} 
\date{\today} 

\newenvironment{qparts}{\begin{enumerate}[{(}a{)}]}{\end{enumerate}} %定义了一个新的环境qparts,用于创建带有小括号标记的题目部分。
%\def\endSolutionsmark{$\mathcal{X} \mathcal{I} \mathcal{U} $} %定义了证明结束标记为一个方框符号。
\newenvironment{Solutions}{\par{\bf Solutions}:}%{\endSolutionsmark\smallskip} %定义了一个新的环境Solutions,用于书写数学证明。


\lstdefinestyle{javastyle}{ %%定义了一个样式mystyle,指定了Java语言,设定了代码的基本样式、注释样式、关键词样式等等
    language=Java,
    basicstyle=\ttfamily\small, % 设置代码字体为大一些
    commentstyle=\color{OliveGreen},
    keywordstyle=\color{RoyalBlue},
    numberstyle=\tiny\color{gray},
    numbers=left,
    frame=single,
    breaklines=true,
    breakatwhitespace=true,
    tabsize=4
}

\lstdefinestyle{cppstyle}{
    language=C++,
    basicstyle=\ttfamily\small,
    commentstyle=\color{OliveGreen},
    keywordstyle=\color{RoyalBlue},
    numberstyle=\tiny\color{gray},
    numbers=left,
    frame=single,
    breaklines=true,
    breakatwhitespace=true,
    tabsize=4
}

\textheight=9in
\textwidth=6.5in
\topmargin=-.75in
\oddsidemargin=0.25in
\evensidemargin=0.25in %设置了页面的尺寸和边距。


%\lstset{style=mystyle}

\begin{document}
\maketitle

%Collaborators: PUT SOMETHING HERE (LIST OF YOUR COLLABORATORS, OR WRITE NONE)

\begin{abstract}
    %Abstract goes here...
    二叉树实现及应用 面向过程 面向对象

    定义
    
    二叉树遍历( 九种方法 递归 非递归 * 2 * 3 

    preorder inorder postorder 

    复杂度 时间空间

    最坏情况 最好情况 最废空间 最省空间

    二叉树的主要性质

    语法制导编辑器

    修改数据结构 (不借助外力 stack queue 

    三叉链表 利用空指针(增加标识

    哈夫曼树 定义性质应用

    二叉树分类

    BST

    ASL 平均查找长度
    
\end{abstract}













\section*{1. 二叉树实现}
\begin{qparts}
    \item 
    功能实现
    
    \item
    应用 
\end{qparts}

\subsection*{1.1 面向过程}

\begin{Solutions}
\begin{enumerate} % for numbers

\item 结构体
ArrayDeque.java
\begin{lstlisting}
/** Array based list.
*  @author zilong
*/


\end{lstlisting}


\end{enumerate}

\end{Solutions}


\subsection*{1.2 面向对象 }

\begin{Solutions}

\begin{enumerate}
\item interface 
BinaryTree.java

\begin{lstlisting}[style=cppstyle]
    #include <iostream>
    #include "../hw2/SeqStack.cpp"
    //#define NULL nullptr
    #define cin std::cin 
    #define cout std::cout
    #define endl std::endl
    
    
    template <class T>
    class BinaryTreeNode{
    private:
        T data;           			
        BinaryTreeNode<T>*  left; 	
        BinaryTreeNode<T>*  right; 	
    
    
    public:
        BinaryTreeNode(); 	                   
        BinaryTreeNode(const T& elem);	
        BinaryTreeNode(const T& elem, BinaryTreeNode<T>* l, BinaryTreeNode<T>* r);
        ~BinaryTreeNode(){};
        T value() const;				
        BinaryTreeNode<T>*  leftchild() const;	
        BinaryTreeNode<T>*  rightchild() const;	
        void  setLeftchild(BinaryTreeNode<T>*);	
        void  setRightchild(BinaryTreeNode<T>*);	
        void  setValue(const T& val);		
        bool  isLeaf() const;		
    };
    
    
    
    template<class T>
    BinaryTreeNode<T>::BinaryTreeNode(){
        data = nullptr;
        left = nullptr;
        right = nullptr;
    }
    
    template<class T>
    BinaryTreeNode<T>::BinaryTreeNode(const T& elem){
        data = elem;
        left = nullptr;
        right = nullptr;
    }
    
    template<class T>
    BinaryTreeNode<T>::BinaryTreeNode(const T& elem, BinaryTreeNode<T>* l, BinaryTreeNode<T>* r){
        data = elem;
        left = l;
        right = r;
    }
    
    
    template<class T>
    T BinaryTreeNode<T>::value() const{
        return this->data;
    }
    
    template<class T>
    BinaryTreeNode<T>*  BinaryTreeNode<T>::leftchild() const{
        return this->left;
    }
    
    template<class T>
    BinaryTreeNode<T>*  BinaryTreeNode<T>::rightchild() const{
        return this->right;
    }
    
    template<class T>
    void BinaryTreeNode<T>::setLeftchild(BinaryTreeNode<T>* Lchild){
        this->left = Lchild;
    }
    
    template<class T>
    void BinaryTreeNode<T>::setRightchild(BinaryTreeNode<T>* Rchild){
        this->right = Rchild;
    }
    
    template<class T>
    void BinaryTreeNode<T>::setValue(const T& Value){
        this->data = Value;
    }
    
    template<class T>
    bool BinaryTreeNode<T>::isLeaf() const{
        return false;
    }
    
    
    
    
    
    template <class T>
    class BinaryTree{
    protected:
        BinaryTreeNode<T>* root;        			
    public:
        BinaryTree() {root = nullptr;} 			
        BinaryTree(BinaryTreeNode<T>* r) {root = r;}
        ~BinaryTree() { DeleteBinaryTree(root); };    	
        bool isEmpty() { return root==nullptr; };  		
        void visit(BinaryTree<T>& curr){cout <<curr->data << " ";} 
        BinaryTreeNode<T>*& Root() {return root;};   	
        void CreateTree(const T& data, BinaryTreeNode<T>* lefttree, BinaryTreeNode<T>* righttree);
        void CreateTree(BinaryTreeNode<T> *&r); 
        void DeleteBinaryTree(BinaryTreeNode<T>* root); 
    
        void PreOrder(BinaryTreeNode<T>* root); 	
        void InOrder(BinaryTreeNode<T>* root); 	
        void PostOrder(BinaryTreeNode<T>* root); 	
        void PreOrderWithoutRecusion(BinaryTreeNode<T>* root);
        void InOrderWithoutRecusion(BinaryTreeNode<T>* root); 
        void PostOrderWithoutRecusion(BinaryTreeNode<T>* root);
        void LevelOrder(BinaryTreeNode<T>* root);	 
    };
    
    
    template<class T> 
    void BinaryTree<T>::CreateTree(const T& data, BinaryTreeNode<T>* leftTree, BinaryTreeNode<T>* rightTree) {
        root = BinaryTreeNode<T>(data, leftTree, rightTree);
        BinaryTree(root);
    }
    
    
    
    
    template<class T> 
    void BinaryTree<T>::DeleteBinaryTree(BinaryTreeNode<T>* Root){
        if (Root != NULL) {
            DeleteBinaryTree(Root->left);
            DeleteBinaryTree(Root->right);  
            delete Root;                    
        }
    }
    
    template<class T>
    void BinaryTree<T>::PreOrder(BinaryTreeNode<T>* root){
        if (root == NULL)  return;
        visit(root->value());			
        PreOrder(root->leftchild());		
        PreOrder(root->rightchild());		
    }
    
    
    template<class T>
    void BinaryTree<T>::PreOrderWithoutRecusion(BinaryTreeNode <T> * root){
        SeqStack<BinaryTreeNode<T>* > tStack;
        BinaryTreeNode<T>* pointer = root;
        while(!tStack.empty() || pointer){
            if (pointer){
                visit(pointer->value());               
                tStack.push(pointer);                   
                pointer = pointer->leftchild();     
            } else{	                                    
                pointer = tStack.top();              
                tStack.pop();                          
                pointer = pointer->rightchild(); }
        }
    }
\end{lstlisting}

\end{enumerate}

\end{Solutions}






\clearpage

\newpage

%\iffalse

\section*{2.二叉树遍历}
\begin{qparts}
\item 

二叉树遍历( 九种方法 递归 非递归 * 2 * 3 

preorder inorder postorder 

复杂度 时间空间

最坏情况 最好情况 最废空间 最省空间


\end{qparts}

\subsection*{preorder}

\subsubsection*{recursive}
\begin{Solutions}
    \begin{lstlisting}
        
    \end{lstlisting}
\end{Solutions}

\subsubsection*{nonrecursive}
\begin{enumerate}
    \item recursivelike
    \begin{Solutions}
    
    \end{Solutions}
    

    
    \item other way 
    \begin{Solutions}
    
    \end{Solutions}
    
\end{enumerate}


\subsection*{inorder}
\subsubsection*{recursive}
\begin{Solutions}
    
\end{Solutions}
\subsubsection*{nonrecursive}
\begin{enumerate}
    \item recursivelike
    \begin{Solutions}
    
    \end{Solutions}
    

    
    \item other way 
    \begin{Solutions}
    
    \end{Solutions}
    
\end{enumerate}

\subsection*{postorder}
\subsubsection*{recursive}
\begin{Solutions}
    
\end{Solutions}
\subsubsection*{nonrecursive}
\begin{enumerate}
    \item recursivelike
    \begin{Solutions}
    
    \end{Solutions}
    

    
    \item other way 
    \begin{Solutions}
    
    \end{Solutions}
    
\end{enumerate}

\subsection*{复杂度分析}
\begin{enumerate}
    \item   复杂度 时间空间
    \item   最坏情况 最好情况 %/最废空间 最省空间
\end{enumerate}


\subsection*{层次遍历}
\begin{enumerate}
    \item   复杂度 时间空间
    \item   借助外力 stack queue 逻辑
\end{enumerate}









\newpage
\section*{3.二叉树的定义、主要性质、定理}
\begin{qparts}
    \item 递归定义及基本术语
    \item 分类 顺序存储 链式存储
    \item 性质*5
\end{qparts}

\subsection*{definition}
\begin{Solutions}
    \begin{enumerate}
        \item   a recursive definition
    \end{enumerate}


\end{Solutions}

\subsection*{category}
\begin{Solutions}

    \begin{enumerate}
        \item sequential structure
        \item list structure
    \end{enumerate}

\end{Solutions}


\subsection*{properties quality character}
\begin{Solutions}
    \begin{enumerate}
    \item i \& $2^ {i-1}$ floor
    \item k \& $2^k - 1$ all the tree 
    \item $ n_0 = n_2 + 1 $
    \item height = $[\log _2 n ] + 1$
    \item structure of root, Lchild, Rchild 
    \end{enumerate}
    
\end{Solutions}






\newpage
\section*{4.证明}
YOUR ANSWER GOES HERE

\subsection*{something to prove}
\begin{Solutions}
    \begin{enumerate}
        \item 给定tree 遍历序列唯一 给定位置唯一存在( 顺序唯一
        \item 中序遍历猜想 + 先序猜想
        \item 遍历时的性质*5
        \item n node  2*n pointer  N-1 in use 
    \end{enumerate}

\end{Solutions}

\newpage
\section*{5.修改数据结构 }
YOUR ANSWER GOES HERE
\begin{qparts}
    \item 
    利用空指针(增加标识
    
    \item
    三叉链表 
\end{qparts}

\subsection*{1.1 利用空指针}

\begin{Solutions}
\begin{enumerate} % for numbers

\item 结构体
ArrayDeque.java
\begin{lstlisting}
/** Array based list.
*  @author zilong
*/


\end{lstlisting}


\end{enumerate}

\end{Solutions}


\subsection*{1.2 三叉链表 }

\begin{Solutions}

\begin{enumerate}
\item interface 
List.java

\begin{lstlisting}
public interface List<Item> 
\end{lstlisting}

\end{enumerate}

\end{Solutions}






\clearpage



\newpage


\newpage
\section*{6.Huffman tree}

\begin{qparts}
    \item 递归定义及基本术语
    \item 分类 顺序存储 链式存储
    \item 应用*4
\end{qparts}

\subsection*{definition}
\begin{Solutions}
    \begin{enumerate}
        \item   a recursive definition
    \end{enumerate}


\end{Solutions}

\subsection*{category}
\begin{Solutions}

    \begin{enumerate}
        \item sequential structure
        \item list structure
    \end{enumerate}

\end{Solutions}


\subsection*{properties quality character}
\begin{Solutions}
    \begin{enumerate}
    \item i \& $2^ {i-1}$ floor
    \item k \& $2^k - 1$ all the tree 
    \item $ n_0 = n_2 + 1 $
    \item height = $[\log _2 n ] + 1$
    \item structure of root, Lchild, Rchild 
    \end{enumerate}
    
\end{Solutions}



\newpage
\section*{7.Binary searching/sorting Tree }


\newpage
\section*{8.BinaryTree categories}


\newpage
\section*{9.BinaryTree in java}

%\fi
\end{document}
