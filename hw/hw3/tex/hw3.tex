
\documentclass[UTF8]{ctexart}
%\documentclass[11pt]{article} %指定文档的类型和基本格式。这里选择了article类,字体大小为11磅。
\usepackage{amsmath,textcomp,amssymb,geometry,graphicx,enumerate} %加载了一些宏包,这些宏包提供了额外的功能和格式设定,例如数学符号、文本特殊符号、排版布局等。

\usepackage{listings}
\usepackage[dvipsnames]{xcolor}

%\usepackage[utf8]{inputenc}
%\usepackage{xeCJK} % Added for Chinese support
%\setCJKmainfont{SimSum} % Set the Chinese font, you can change it to any font you have


%\def\Name{zilong}
\def\Name{王子隆}  
\def\SID{2221411126}  
\def\Homework{3} 
\def\Session{Autumn 2023} 


\title{DS--Autumn 2023 --- Homework \Homework Solutions} 
\author{\Name, SID \SID} 
\markboth{DS--\Session\  Homework \Homework\ \Name}{DS--\Session\ Homework \Homework\ \Name} 
\pagestyle{myheadings} 
\date{\today} 

\newenvironment{qparts}{\begin{enumerate}[{(}a{)}]}{\end{enumerate}} %定义了一个新的环境qparts,用于创建带有小括号标记的题目部分。
%\def\endSolutionsmark{$\mathcal{X} \mathcal{I} \mathcal{U} $} %定义了证明结束标记为一个方框符号。
\newenvironment{Solutions}{\par{\bf Solutions}:}%{\endSolutionsmark\smallskip} %定义了一个新的环境Solutions,用于书写数学证明。


\lstdefinestyle{mystyle}{ %%定义了一个样式mystyle,指定了Java语言,设定了代码的基本样式、注释样式、关键词样式等等
    language=Java,
    basicstyle=\ttfamily\small, % 设置代码字体为大一些
    commentstyle=\color{OliveGreen},
    keywordstyle=\color{RoyalBlue},
    numberstyle=\tiny\color{gray},
    numbers=left,
    frame=single,
    breaklines=true,
    breakatwhitespace=true,
    tabsize=4
}


\textheight=9in
\textwidth=6.5in
\topmargin=-.75in
\oddsidemargin=0.25in
\evensidemargin=0.25in %设置了页面的尺寸和边距。


\lstset{style=mystyle}

\begin{document}
\maketitle

%Collaborators: PUT SOMETHING HERE (LIST OF YOUR COLLABORATORS, OR WRITE NONE)

\begin{abstract}
    %Abstract goes here...
    二叉树实现及应用 面向过程 面向对象

    定义
    
    二叉树遍历( 九种方法 递归 非递归 * 2 * 3 

    preorder inorder postorder 

    复杂度 时间空间

    最坏情况 最好情况 最废空间 最省空间

    二叉树的主要性质

    语法制导编辑器

    修改数据结构 (不借助外力 stack queue 

    

\end{abstract}













\section*{1. 二叉树实现}
\begin{qparts}
    \item 
    功能实现
    
    \item
    应用 
\end{qparts}

\subsection*{1.1 面向过程}

\begin{Solutions}
\begin{enumerate} % for numbers

\item 结构体
ArrayDeque.java
\begin{lstlisting}
/** Array based list.
*  @author zilong
*/


\end{lstlisting}


\end{enumerate}

\end{Solutions}


\subsection*{1.2 面向对象 }

\begin{Solutions}

\begin{enumerate}
\item interface 
List.java

\begin{lstlisting}
public interface List<Item> 
\end{lstlisting}

\end{enumerate}

\end{Solutions}






\clearpage

\newpage

%\iffalse

\section*{2.二叉树遍历}
\begin{qparts}
\item 

二叉树遍历( 九种方法 递归 非递归 * 2 * 3 

preorder inorder postorder 

复杂度 时间空间

最坏情况 最好情况 最废空间 最省空间


\end{qparts}

\subsection*{preorder}

\subsubsection*{recursive}
\begin{Solutions}
    \begin{lstlisting}
        
    \end{lstlisting}
\end{Solutions}

\subsubsection*{nonrecursive}
\begin{enumerate}
    \item recursivelike
    \begin{Solutions}
    
    \end{Solutions}
    

    
    \item other way 
    \begin{Solutions}
    
    \end{Solutions}
    
\end{enumerate}


\subsection*{inorder}
\subsubsection*{recursive}
\begin{Solutions}
    
\end{Solutions}
\subsubsection*{nonrecursive}
\begin{enumerate}
    \item recursivelike
    \begin{Solutions}
    
    \end{Solutions}
    

    
    \item other way 
    \begin{Solutions}
    
    \end{Solutions}
    
\end{enumerate}

\subsection*{postorder}
\subsubsection*{recursive}
\begin{Solutions}
    
\end{Solutions}
\subsubsection*{nonrecursive}
\begin{enumerate}
    \item recursivelike
    \begin{Solutions}
    
    \end{Solutions}
    

    
    \item other way 
    \begin{Solutions}
    
    \end{Solutions}
    
\end{enumerate}

\subsection*{复杂度分析}
\begin{enumerate}
    \item   复杂度 时间空间
    \item   最坏情况 最好情况 %/最废空间 最省空间
\end{enumerate}


\subsection*{层次遍历}
\begin{enumerate}
    \item   复杂度 时间空间
    \item   借助外力 stack queue 逻辑
\end{enumerate}









\newpage
\section*{3.二叉树的定义、主要性质、定理}
\begin{qparts}
    \item 递归定义及基本术语
    \item 分类 顺序存储 链式存储
    \item 性质*5
\end{qparts}

\subsection*{definition}
\begin{Solutions}
    \begin{enumerate}
        \item   a recursive definition
    \end{enumerate}


\end{Solutions}

\subsection*{category}
\begin{Solutions}

    \begin{enumerate}
        \item sequential structure
        \item list structure
    \end{enumerate}

\end{Solutions}


\subsection*{properties quality character}
\begin{Solutions}
    \begin{enumerate}
    \item i \& $2^ {i-1}$ floor
    \item k \& $2^k - 1$ all the tree 
    \item $ n_0 = n_2 + 1 $
    \item height = $[\log _2 n ] + 1$
    \item structure of root, Lchild, Rchild 
    \end{enumerate}
    
\end{Solutions}






\newpage
\section*{4.证明}
YOUR ANSWER GOES HERE

\subsection*{something to prove}
\begin{Solutions}
    \begin{enumerate}
        \item 给定tree 遍历序列唯一 给定位置唯一存在( 顺序唯一
        \item 中序遍历猜想 + 先序猜想
        \item 遍历时的性质*5
        \item n node  2*n pointer  N-1 in use 
    \end{enumerate}

\end{Solutions}

\newpage
\section*{5.修改数据结构 }
YOUR ANSWER GOES HERE


\newpage
\section*{6.BinaryTree in java}

%\fi
\end{document}
