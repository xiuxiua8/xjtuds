
%\documentclass[11pt]{article} %指定文档的类型和基本格式。这里选择了article类,字体大小为11磅。
\documentclass[UTF8]{ctexart}
\usepackage{amsmath,textcomp,amssymb,geometry,graphicx,enumerate} %加载了一些宏包,这些宏包提供了额外的功能和格式设定,例如数学符号、文本特殊符号、排版布局等。

\usepackage{listings}
\usepackage[dvipsnames]{xcolor}

%\usepackage[utf8]{inputenc}
%\usepackage{xeCJK} % Added for Chinese support
%\setCJKmainfont{SimSum} % Set the Chinese font, you can change it to any font you have


%\def\Name{zilong}
\def\Name{王子隆}  % Your name 定义了一个名为\Name的变量,它的内容是"PUT SOMETHING HERE"。通常这个变量会在文档中被调用以插入相应的信息。
\def\SID{2221411126}  % Your student ID number 定义了一个名为\SID的变量,它的内容也是"PUT SOMETHING HERE"。
\def\Lab{2} % Number of Lab 定义了一个名为\Lab的变量,它的内容是"n",表示编号。
\def\Session{Autumn 2023} %定义了一个名为\Session的变量,它的内容是"Autumn 2023",表示学期。


\title{DS--Autumn 2023 --- Lab \Lab    Solutions} %设置了文档的标题,包括课程名、学期和作业编号。
\author{\Name, SID \SID} %设置了文档的作者信息,使用了之前定义的\Name和\SID变量。
\markboth{DS--\Session\  Lab \Lab\ \Name}{DS--\Session\ Lab \Lab\ \Name} %设置了页眉的内容。
\pagestyle{myheadings} %设置了页眉的内容。
\date{\today} %清空了默认的日期显示。

\newenvironment{qparts}{\begin{enumerate}[{(}a{)}]}{\end{enumerate}} %定义了一个新的环境qparts,用于创建带有小括号标记的题目部分。
%\def\endSolutionsmark{$\mathcal{X} \mathcal{I} \mathcal{U} $} %定义了证明结束标记为一个方框符号。
\newenvironment{Solutions}{\par{\bf Solutions}:}%{\endSolutionsmark\smallskip} %定义了一个新的环境Solutions,用于书写数学证明。


\lstdefinestyle{javastyle}{ %%定义了一个样式mystyle,指定了Java语言,设定了代码的基本样式、注释样式、关键词样式等等
    language=Java,
    basicstyle=\ttfamily\small, % 设置代码字体为大一些
    commentstyle=\color{OliveGreen},
    keywordstyle=\color{RoyalBlue},
    numberstyle=\tiny\color{gray},
    numbers=left,
    frame=single,
    breaklines=true,
    breakatwhitespace=true,
    tabsize=4
}

\lstdefinestyle{cppstyle}{
    language=C++,
    basicstyle=\ttfamily\small,
    commentstyle=\color{OliveGreen},
    keywordstyle=\color{RoyalBlue},
    numberstyle=\tiny\color{gray},
    numbers=left,
    frame=single,
    breaklines=true,
    breakatwhitespace=true,
    tabsize=4
}

\textheight=9in
\textwidth=6.5in
\topmargin=-.75in
\oddsidemargin=0.25in
\evensidemargin=0.25in %设置了页面的尺寸和边距。



\begin{document}
\maketitle

%Collaborators: PUT SOMETHING HERE (LIST OF YOUR COLLABORATORS, OR WRITE NONE)

\begin{abstract}
    %Abstract goes here...

    1.问题描述

    一个农夫带着一只狼、一只羊和一棵白菜,身处河的南岸。他要把这些东西全部运到北岸。
    他面前只有一条小船,船只能容下他和一件物品,另外只有农夫才能撑船。
    如果农夫在场,则狼不能吃羊,羊不能吃白菜,否则狼会吃羊,羊会吃白菜,
    所以农夫不能留下羊和白菜自己离开,也不能留下狼和羊自己离开,而狼不吃白菜。

    请求出农夫将所有的东西运过河的方案。

    2.实现提示

    求解这个问题的简单方法是一步一步进行试探,每一步搜索所有可能的选择,
    对前一步合适的选择后再考虑下一步的各种方案。
    要模拟农夫过河问题,首先需要对问题中的每个角色的位置进行描述。
    可用4位二进制数顺序分别表示农夫、狼、白菜和羊的位置。
    用0表在南岸,1表示在北岸。例如,整数5 (0101)表示农夫和白菜在南岸,而狼和羊在北岸。
    现在问题变成:从初始的状态二进制0000(全部在河的南岸)出发,寻找一种全部由安全状态构成的状态序列,
    它以二进制1111(全部到达河的北岸)为最终目标。
    总状态共16种(0000到1111),(或者看成16个顶点的有向图)
    可采用广度优先或深度优先的搜索策略---得到从0000到1111的安全路径。

    以广度优先为例:整数队列---逐层存放下一步可能的安全状态;
    Visited[16]数组标记该状态是否已访问过,若访问过,则记录前驱状态值---安全路径。
    最终的过河方案应用汉字显示出每一步的两岸状态。

\end{abstract}







\section*{1. 数据结构实现}
\begin{qparts}
    \item 功能实现


    \item 应用 
    
\end{qparts}



\subsection*{1.1 abc}

\begin{Solutions}
\begin{enumerate} % for numbers
\item 1
ArrayDeque.java
\begin{lstlisting}

\end{lstlisting}

\item 2

\begin{lstlisting}

\end{lstlisting}


\end{enumerate}
\end{Solutions}








\iffalse
\newpage
\section*{2.}
YOUR ANSWER GOES HERE
\fi


\end{document}
