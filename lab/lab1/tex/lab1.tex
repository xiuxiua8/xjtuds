
%\documentclass[11pt]{article} %指定文档的类型和基本格式。这里选择了article类,字体大小为11磅。
\documentclass[UTF8]{ctexart}
\usepackage{amsmath,textcomp,amssymb,geometry,graphicx,enumerate} %加载了一些宏包,这些宏包提供了额外的功能和格式设定,例如数学符号、文本特殊符号、排版布局等。

\usepackage{listings}
\usepackage[dvipsnames]{xcolor}

%\usepackage[utf8]{inputenc}
%\usepackage{xeCJK} % Added for Chinese support
%\setCJKmainfont{SimSum} % Set the Chinese font, you can change it to any font you have


%\def\Name{zilong}
\def\Name{王子隆}  % Your name 定义了一个名为\Name的变量,它的内容是"PUT SOMETHING HERE"。通常这个变量会在文档中被调用以插入相应的信息。
\def\SID{2221411126}  % Your student ID number 定义了一个名为\SID的变量,它的内容也是"PUT SOMETHING HERE"。
\def\Lab{1} % Number of Lab 定义了一个名为\Lab的变量,它的内容是"n",表示编号。
\def\Session{Autumn 2023} %定义了一个名为\Session的变量,它的内容是"Autumn 2023",表示学期。


\title{DS--Autumn 2023 --- Lab \Lab    Solutions} %设置了文档的标题,包括课程名、学期和作业编号。
\author{\Name, SID \SID} %设置了文档的作者信息,使用了之前定义的\Name和\SID变量。
\markboth{DS--\Session\  Lab \Lab\ \Name}{DS--\Session\ Lab \Lab\ \Name} %设置了页眉的内容。
\pagestyle{myheadings} %设置了页眉的内容。
\date{\today} %清空了默认的日期显示。

\newenvironment{qparts}{\begin{enumerate}[{(}a{)}]}{\end{enumerate}} %定义了一个新的环境qparts,用于创建带有小括号标记的题目部分。
%\def\endSolutionsmark{$\mathcal{X} \mathcal{I} \mathcal{U} $} %定义了证明结束标记为一个方框符号。
\newenvironment{Solutions}{\par{\bf Solutions}:}%{\endSolutionsmark\smallskip} %定义了一个新的环境Solutions,用于书写数学证明。


\lstdefinestyle{javastyle}{ %%定义了一个样式mystyle,指定了Java语言,设定了代码的基本样式、注释样式、关键词样式等等
    language=Java,
    basicstyle=\ttfamily\small, % 设置代码字体为大一些
    commentstyle=\color{OliveGreen},
    keywordstyle=\color{RoyalBlue},
    numberstyle=\tiny\color{gray},
    numbers=left,
    frame=single,
    breaklines=true,
    breakatwhitespace=true,
    tabsize=4
}

\lstdefinestyle{cppstyle}{
    language=C++,
    basicstyle=\ttfamily\small,
    commentstyle=\color{OliveGreen},
    keywordstyle=\color{RoyalBlue},
    numberstyle=\tiny\color{gray},
    numbers=left,
    frame=single,
    breaklines=true,
    breakatwhitespace=true,
    tabsize=4
}

\textheight=9in
\textwidth=6.5in
\topmargin=-.75in
\oddsidemargin=0.25in
\evensidemargin=0.25in %设置了页面的尺寸和边距。



\begin{document}
\maketitle

%Collaborators: PUT SOMETHING HERE (LIST OF YOUR COLLABORATORS, OR WRITE NONE)

\begin{abstract}
    %Abstract goes here...
    1.问题描述


    假设有一个能装入总体积为T的背包和n件体积分别为$w_1,w_2,$…$w_n$的物品,能否从n件物品中挑选若干件恰好装满背包,
    即使$w_1+w_2+$…$+w_m=T$,要求找出所有满足上述条件的解。 
   
    例如:当$T=10$,各件物品的体积{1,8,4,3,5,2}时,可找到下列4组解:

     (1,4,3,2)

      (1,4,5)

      (8,2)

      (3,5,2)。

    2.实现提示


    可利用回溯法的设计思想来解决背包问题。
    首先,将物品排成一列,然后,顺序选取物品装入背包,若已选取第i件物品后未满,
    则继续选取第i+1件,若该件物品“太大”不能装入,则弃之,继续选取下一件,直至背包装满为止。

    如果在剩余的物品中找不到合适的物品以填满背包,则说明“刚刚”装入的物品“不合适”,应将它取出“弃之一边”,
    继续再从“它之后”的物品中选取,如此重复,直到求得满足条件的解,或者无解。

    由于回溯求解的规则是“后进先出”,自然要用到“栈”。

    进一步考虑:如果每件物品都有体积和价值,背包又有大小限制,求解背包中存放物品总价值最大的问题解---最优解或近似最优解。


\end{abstract}







\section*{1. 数据结构实现}
\begin{qparts}
    \item 功能实现


    \item 应用 
    
\end{qparts}



\subsection*{1.1 abc}

\begin{Solutions}
\begin{enumerate} % for numbers
\item 1
ArrayDeque.java
\begin{lstlisting}

\end{lstlisting}

\item 2

\begin{lstlisting}

\end{lstlisting}


\end{enumerate}
\end{Solutions}








\iffalse
\newpage
\section*{2.}
YOUR ANSWER GOES HERE
\fi


\end{document}
