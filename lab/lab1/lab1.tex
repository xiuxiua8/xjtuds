
%\documentclass[11pt]{article} %指定文档的类型和基本格式。这里选择了article类,字体大小为11磅。
\documentclass[UTF8]{ctexart}
\usepackage{amsmath,textcomp,amssymb,geometry,graphicx,enumerate} %加载了一些宏包,这些宏包提供了额外的功能和格式设定,例如数学符号、文本特殊符号、排版布局等。

\usepackage{listings}
\usepackage[dvipsnames]{xcolor}

%\usepackage[utf8]{inputenc}
%\usepackage{xeCJK} % Added for Chinese support
%\setCJKmainfont{SimSum} % Set the Chinese font, you can change it to any font you have


%\def\Name{zilong}
\def\Name{王子隆}  % Your name 定义了一个名为\Name的变量,它的内容是"PUT SOMETHING HERE"。通常这个变量会在文档中被调用以插入相应的信息。
\def\SID{2221411126}  % Your student ID number 定义了一个名为\SID的变量,它的内容也是"PUT SOMETHING HERE"。
\def\Homework{1} % Number of Homework 定义了一个名为\Homework的变量,它的内容是"n",表示作业编号。
\def\Session{Autumn 2023} %定义了一个名为\Session的变量,它的内容是"Autumn 2023",表示学期。


\title{DS--Autumn 2023 --- Homework \Homework Solutions} %设置了文档的标题,包括课程名、学期和作业编号。
\author{\Name, SID \SID} %设置了文档的作者信息,使用了之前定义的\Name和\SID变量。
\markboth{DS--\Session\  Homework \Homework\ \Name}{DS--\Session\ Homework \Homework\ \Name} %设置了页眉的内容。
\pagestyle{myheadings} %设置了页眉的内容。
\date{\today} %清空了默认的日期显示。

\newenvironment{qparts}{\begin{enumerate}[{(}a{)}]}{\end{enumerate}} %定义了一个新的环境qparts,用于创建带有小括号标记的题目部分。
%\def\endSolutionsmark{$\mathcal{X} \mathcal{I} \mathcal{U} $} %定义了证明结束标记为一个方框符号。
\newenvironment{Solutions}{\par{\bf Solutions}:}%{\endSolutionsmark\smallskip} %定义了一个新的环境Solutions,用于书写数学证明。


\lstdefinestyle{mystyle}{ %%定义了一个样式mystyle,指定了Java语言,设定了代码的基本样式、注释样式、关键词样式等等
    language=Java,
    basicstyle=\ttfamily\small, % 设置代码字体为大一些
    commentstyle=\color{OliveGreen},
    keywordstyle=\color{RoyalBlue},
    numberstyle=\tiny\color{gray},
    numbers=left,
    frame=single,
    breaklines=true,
    breakatwhitespace=true,
    tabsize=4
}


\textheight=9in
\textwidth=6.5in
\topmargin=-.75in
\oddsidemargin=0.25in
\evensidemargin=0.25in %设置了页面的尺寸和边距。


\lstset{style=mystyle}

\begin{document}
\maketitle

%Collaborators: PUT SOMETHING HERE (LIST OF YOUR COLLABORATORS, OR WRITE NONE)

\begin{abstract}
    %Abstract goes here...
    线性表功能实现及应用
    
    Merge(L1,L2)

    Dispose (L1, 12)

    Sort (L)
  
    Insert(L,a,b) 不/带头链表L,在元素值a之前插入b

    应用 一元多项式的加減法(链表)


    %插入元素通过覆盖元素,并且向后增加节点

    检测环形链表证明链表有无环

    增加bool数据类型判断环形链表是否遍历完毕

    hanoi塔

    数据库(交并差叉积)
    
    半矩阵存储

\end{abstract}













\section*{1. 线性表功能实现及应用}
\begin{qparts}
    \item 
    功能实现
    
    Merge(L1,L2)

    Dispose (L1, 12)

    Sort (L)
  
    Insert(L,a,b) 不/带头链表L,在元素值a之前插入b

    \item
    应用 
    
    一元多项式的加減法(链表)
\end{qparts}

\subsection*{1.1 数组、链表、双向循环链表建立}

\begin{Solutions}
\begin{enumerate} % for numbers

\item 数组
ArrayDeque.java
\begin{lstlisting}
/** Array based list.
*  @author zilong
*/


\end{lstlisting}

\item 双向循环链表
LinkedListDeque.java
\begin{lstlisting}
/** DLLists based list.
*  @author zilong
*/

public class LinkedListDeque<T> implements List<T>{

    

}

\end{lstlisting}

\end{enumerate}

\end{Solutions}


\subsection*{1.2 应用 }

\begin{Solutions}
\begin{enumerate}
\item interface 
List.java
\begin{lstlisting}
public interface List<Item> 
\end{lstlisting}

\end{enumerate}

\end{Solutions}

\newpage

\iffalse
\section*{2.}
\begin{qparts}
\item
YOUR ANSWER GOES HERE

\end{qparts}


\newpage
\section*{3.}
YOUR ANSWER GOES HERE


\newpage
\section*{4.}
YOUR ANSWER GOES HERE


\newpage
\section*{5.}
YOUR ANSWER GOES HERE


\newpage
\section*{6.}
YOUR ANSWER GOES HERE

\fi
\end{document}
